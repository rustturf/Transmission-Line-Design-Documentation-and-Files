\documentclass[a4paper,11pt]{article}
\usepackage[utf8]{inputenc}
\usepackage{graphicx}
\usepackage{amsmath}
\usepackage{amssymb}
\graphicspath{{Images/}}
\usepackage{wrapfig}
\usepackage{titling}
\usepackage{caption}
\usepackage{float}
\usepackage{bigstrut}
\usepackage[left=1in,right=1in,top=1in,bottom=1in,paper=a4paper]{geometry}
\usepackage{tocloft}
\usepackage{setspace}
\setlength{\parindent}{0pt}
\renewcommand{\contentsname}{\Huge{Contents}}
\renewcommand{\cftsecfont}{ \Large \bfseries}
\renewcommand{\cftsubsecfont}{\large}
\renewcommand{\cftaftertoctitle}{\hfill\\\vspace{5pt}}
\usepackage{array,multirow}
\setlength{\tabcolsep}{20pt} % Gap before text starts
\renewcommand{\arraystretch}{2} % Cell height scalings
\begin{document}
\begin{center}
\vspace{1cm}
\includegraphics[scale=1.3]{Figures/KhCE.png}\\
\end{center}
\vspace{0.2cm}
\begin{center}
    \Large {Tribhuvan University}
\end{center}
\begin{center}
    \huge{\textbf{KHWOPA COLLEGE OF ENGINEERING}}\\ \vspace{0.5cm}
    \centering \small An Undertaking of Bhaktapur Municipality \\
    Libali-8, Bhaktapur\\
    Nepal
\end{center}
\vspace{1cm}
\begin{center}
\Large{\textbf{A Report On}}\\
\vspace{0.6cm}
\textbf{\huge{Transmission and Distribution Design}}
\end{center}
\vspace{0.5cm}
\begin{center}
    \textbf{\Large{\underline{Submitted By:}}}\\[0.4cm]
    \textbf{\Large{Shubhanga Aryal}}\\[0.3cm]
    \textbf{\Large{KCE078BEL038}}
\end{center}
\vspace{1cm}
\begin{center}
    \textbf{\Large{\underline{Submitted To:}}}\\[0.4cm]
    \textbf{\Large{Er. Shreeshuva Maharjan}}\\[0.3cm]
    \textbf{\Large{\ \ \ Er. Anil Bhatt}}
\end{center}
\vspace{1cm}
\begin{center}
\textbf{\Large{Department of Electrical Engineering}}\\
\vspace{0.3cm}
\textbf{{KHWOPA COLLEGE OF ENGINEERING}}\\
\vspace{0.3cm}
\textbf{{Libali, Bhaktapur}}
\end{center}
\thispagestyle{empty}
\newpage
\doublespacing
\tableofcontents
\thispagestyle{empty}
\newpage
\noindent
\setcounter{page}{1}
\begin{center}
    \huge{\textbf{Transmission And Distribution Design}}
\end{center}
\vspace*{1cm}
\noindent
\onehalfspacing
\large{Design a transmission and distribution design that can transmit a power of $133MW$ for a length of $110 Km$}.
\section{Calculation of the most Economical voltage}
For a given power transmission, increasing the voltage reduces the current which reduces the overall copper loss and energy losses. Lower energy losses means less energy is wasted which allows us to use conductors with smaller cross sectional area, which improves conductor economy. However, transmitting at higher voltages requires better insulation, larger clearances, longer insulators which increases the insulator and construction cost.\\[0.1cm]
To mitigate this issue, the most economical voltage is chosen. The most economical voltage can be defined as the voltage at which the reduction in power and energy loss and conductor cost just balances the increase in insulator and line construction cost, which gives us a minimum total annual cost for generating voltage. The empirical formula for the most economical voltage is given by:
\begin{align}
    V=5.5\sqrt{\dfrac{L}{1.6}+\dfrac{P\times1000}{150\times N_c\times cos\phi}}(in\ KV)
    \label{equation1}
\end{align}
where, $L$ is the length of the transmission line in $km$, $P$ is the power transmitted in $MW$, $N_c$ is the number of circuit in the transmission line and $cos\phi$ is the power factor of the transmission line.\\[0.1cm]
Since the most commonly used number of circuits($N_c$) are 1 and 2, we now calculate the most economical voltage for $N_c=1$ and $N_c=2$.\\[0.1cm]
For $Nc=1$:\\[0.1cm]
Taking the length and Power transmitted as given in the question and substituting the values in equation \ref{equation1} and assuming power factor to be 0.9, we get
\begin{align*}
    V_{eco1}&=5.5\sqrt{\dfrac{110}{1.6}+\dfrac{133\times1000}{150\times 1\times 0.9}}\\[0.1cm]
    &=178.55\ KV
\end{align*}
Similarly, for $N_c=2$:
\begin{align*}
    V_{eco2}&=5.5\sqrt{\dfrac{110}{1.6}+\dfrac{133\times1000}{150\times 2\times 0.9}}\\[0.1cm]
    &=130.3096\ KV
\end{align*}
\newpage
\section{Selection of Economical voltage Near the Standard Transmission Line Voltage}
The most economical voltage can or may not be exactly the voltage that is used in standard transmission line design. Hence, we must choose the nearest standard voltage . The standard voltages used in transmission design are as follows: $\mathbf{55\,\mathrm{kV},\ 132\,\mathrm{kV},\ 220\,\mathrm{kV},\ 400\,\mathrm{kV}}$
. Since the economical voltage $V_{eco1}$ is nearest to $220\ KV$ and $V_{eco2}$ is nearest to $132\ KV$, we choose $V_{eco1}=220\ KV$ and $V_{eco2}=132\ KV$.
\section{Technical Analysis}
To do the technical analysis of the transmission line, we must follow these three steps. They are as follows: 
\begin{itemize}
    \item Calculate multiplying factor($m_{funit}$) for a given length. 
    \item Calculate Surge Impeadance Loading(SIL) for both the cases($N_c=1 \  and\ N_c=2$). 
    \item Calculate the $m_f$ value for both the cases. 
\end{itemize}
The table to analyze the length and its corresponding limit of the multiplying factor is as shown below: 
% Table generated by Excel2LaTeX from sheet 'Sheet1'
\begin{table}[htbp]
  \centering
  \caption{$m_{flimit}$ for various lengths}
    \begin{tabular}{|c|c|}
    \hline
    Length(km) & $m_{flimit}$ \bigstrut\\
    \hline
    80    & 2.75 \bigstrut\\
    \hline
    160   & 2.25 \bigstrut\\
    \hline
    240   & 1.75 \bigstrut\\
    \hline
    320   & 1.35 \bigstrut\\
    \hline
    480   & 1 \bigstrut\\
    \hline
    640   & 0.75 \bigstrut\\
    \hline
    \end{tabular}%
  \label{tab:addlabel}%
\end{table}%
Since the length is 110 Km, which is not available in the table given , we must interpolate the values to find the corresponding values. By using interpolation,   
\begin{align}
    y-y_1&=\dfrac{y_2-y_1}{x_2-x_1}(x-x_1)
\end{align}
Taking the values of $m_{flimit}$ column in y and Length column in x and putting it in equation 2:
\begin{align*}
     y-2.75&=\dfrac{2.25-2.75}{160-80}(110-80)\\[0.1cm]
     y&=2.5625    
\end{align*}
\(\therefore\) $m_{flimit}$ for 110 $km$ is 2.5625.
We know, 
\begin{align}
    m_{flimit}&=\dfrac{P_{capability}}{SIL}=2.5625
\end{align}
We now calculate the surge impedance loading for both the circuits, For $N_c=1 \ and \ Nc=2$. We know, 
\begin{align}
    SIL=\dfrac{(V_{eco})^2}{Z_{nc}}
    \label{equation 4}
\end{align}
where $Z_{nc}$ is the impedance of the number of circuits(400 for $N_c$=1 and 200 for $N_c=2$. Substituting the values in equation \ref{equation 4}:
\begin{align*}
    SIL_1(N_c=1)&=\dfrac{(V_{eco1})^2}{Z_{nc1}}=\dfrac{220^2}{400}=121\ MW\\[0.1cm]
    SIL_2(N_c=2)&=\dfrac{(V_{eco2})^2}{Z_{nc2}}=\dfrac{132^2}{200}=87.12\ MW
\end{align*}
Now we calculate the multiplying factor for both the circuit:
\begin{align*}
    m_{f1(N_{c1})}&=\dfrac{P_{transfer}}{SIL_1}=\dfrac{133}{121}=1.099\\
    m_{f2(N_{c2})}&=\dfrac{P_{transfer}}{SIL_2}=\dfrac{133}{87.12}=1.5266\\    
\end{align*}
Hence the following can be analyzed with the help of table as shown: 
% Table generated by Excel2LaTeX from sheet 'Sheet1'
\begin{table}[!htbp]
  \centering
  \caption{Comparison between the multiplying factor and its limit for both type of circuit}
    \begin{tabular}{|c|c|c|c|c|}
    \hline
    $N_c$    & $V_{eco}$  & $SIL(MW)$ & $m_f$    & $m_{flimit}$ \bigstrut\\
    \hline
    1     & 220   & 121   & 1.099 & 2.5625 \bigstrut\\
    \hline
    2     & 132   & 87.12 & 1.526 & 2.5625 \bigstrut\\
    \hline
    \end{tabular}%
  \label{tab:addlabel}%
\end{table}
\newpage
It can be seen from the table that the value of $m_f$ is less than $m_{flimit}$ for both the circuits. And hence it is acceptable. Now we calculate the margin of $m_f$ for both the circuits. Hence, 
\begin{align*}
    m_{f\ margin(N_{c1})}&=|m_{flimit-m_{f1(N_{c1})}}|=|2.5625-1.099|=1.466\\
    m_{f\ margin(N_{c2})}&=|m_{flimit-m_{f2(N_{c2})}}|=|2.5625-1.526|=1.0372\\
\end{align*}
The value of mf margin is minimum for $N_c=2$. Hence we choose the most economical voltage of 132 KV as our main voltage(V) and move forward with our calculations. 
\section{Air Clearance Calculation }
For the transmission of power, electrical energy is transmitted using transmission
towers and conductors. Therefore, the transmission towers must be properly designed
to ensure safe and reliable operation of the transmission line. For this purpose,
various geometrical and electrical parameters of the tower need to be determined.
The following parameters are required for the tower design:

\[
\begin{aligned}
a  & = \text{minimum air clearance} \\
l  & = \text{length of string} \\
c_l & = \text{cross arm length} \\
b  & = \text{width of the body} \\
c  & = \text{total width of the tower} \\
d  & = \text{length of earth wire from uppermost conductor} \\
\phi & = \text{swing angle} \\
\theta & = \text{shielding angle}
\end{aligned}
\]

These parameters are selected to maintain the required clearances and structural
dimensions of the tower under different operating conditions. \\[0.2cm]
The minimum air clearance between live conductors and between conductor and earth is determined by the maximum operating voltage and safety requirements. It is given by

\[
a = (1 \text{ cm per kV of maximum voltage}) + \text{factor of safety}
\]

The factor of safety accounts for overvoltages, atmospheric variations, and non-uniform electric field effects. A value of \(1.1\) is used to consider the Ferranti effect for transmission lines up to \(220~\text{kV}\), where the receiving-end voltage may rise under light-load or no-load conditions. For higher voltage lines, up to \(400~\text{kV}\), the factor of safety is reduced to \(1.05\) due to improved  design. This ensures sufficient air clearance to avoid flashover and maintain reliable and safe operation of the transmission line.
\subsection{Calculation}
\subsubsection{Minimum Air Clearance}
\begin{align}
    a&=\left(\dfrac{V}{\sqrt3}\times 1.1\times \sqrt2\right)+20cm\\[0.1cm]
    &=\left(\dfrac{132}{\sqrt3}\times 1.1\times \sqrt2
    \right)+20cm\notag\\[0.1cm]
    &=139.5cm\notag
\end{align}
\subsubsection{Length of cross arm($c_l$)}
\begin{align}
    c_l&=2a\\
    &=2\times 138.55\ cm=277.110\ cm\notag
\end{align}
\subsubsection{Tower width(b)}
\begin{align*}
    b&=2a\\
    &=277.110\ cm\notag
\end{align*}
\subsubsection{Insulator string length(l)}
\begin{align}
    l=\sqrt{2}a=195.93cm
\end{align}
\subsubsection{Horizontal separation between conductor(c)}
It is given by:
\begin{align}
    c&=b+2c_l\\
    &=277.11+2\times277.11\notag \\
    &=831.33cm 
\end{align}
\subsubsection{Vertical separation between conductors(y)}
\begin{align}
    y&=\dfrac{l+a}{\sqrt{1-\dfrac{(l+a)^2}{c_l^2}\times(\dfrac{x^2}{y^2})}}\\
    &=365.35cm\notag
\end{align}
The value of the ratio x/y should be taken between 1/4 and 2/3. i.e, $\dfrac{1}{4}<\dfrac{x}{y}<1/3$. Taking the ratio to be 1/3 we get, y=464.215cm. 
\subsubsection{Height of earth wire from topmost conductor for double earth wire}
Since we are using a double circuit line and not single circuit, the earth wire we will use is double circuit wire and not single circuit. \\
For single earth wire, 
\begin{align}
    d&=\sqrt{3}(\dfrac{b}{2}+a)\\
    &=719.95cm\notag
\end{align}
For double earth wire, 
\begin{align}
    d&=\sqrt{3}\times c_l\\
    &=479.96 cm\notag
\end{align}
\section{Number of Insulator Disc Selection}
For a 132 KV disc insulators the disc taken is a $(254\times154)mm^2$. The following parameters are assumed before conducting the tests: 
\[
\begin{aligned}
    \text{Flashover Withstand Ratio}(FWR)&=1.2\\
    \text{Non Standard Atmospheric Condition}(NAC)&=1.1\\
    \text{Factor of safety}(FS)&=1.2\\
    \text{Switching Surge Ratio}(SSR)&=2.8\\
    \text{Switching Impulse Ratio}(SIR)&=1.2\\
    \text{Earthing Factor}&=0.8
\end{aligned}
\]
\subsection{Continuous Operating Voltage}
\begin{itemize}
    \item 1 min dry equivalent withstand test
    \begin{align}
        &=\textit{1 min dry withstand voltage(from table)}\times FWR \times NAC \times FS\\
        &=265*1.15*1.1*1.2 \notag\\
        &=402.27KV \notag
    \end{align}
    According to the table, the number of discs required =7
    \item 1 min wet equivalent withstand test
    \begin{align}
        &=\textit{1 min wet withstand voltage(from table)}\times FWR \times NAC \times FS\\
        &=230*1.15*1.1*1.2 \notag\\
        &=349.14KV \notag
    \end{align}
    According to the table, the number of discs required =9
\end{itemize}
\subsection{Temporary Overvoltage Withstand test($V_{temp}$)}
\begin{align}
&=EF*\textit{maximum system voltage(L-L)*FWR*NAC*FS}\\
&=0.8* \sqrt{2}*145*1.15*1.1*1.2\notag \\
&=249.026 KV\notag
\end{align}
According to the table, no of disc required=6
\subsection{Lightning OverVoltage Withstand Test}
\begin{align}
    &=\textit{Impulse Withstand(from table)}*FWR*NAC*FS\\
    &=550*1.15*1.1*1.2\notag \\
    &=834.9 KV\notag
\end{align}
According to the table, number of discs required=9.
\subsection{Switching OverVoltage Withstand Test }
\begin{align}
    &=\textit{Switching overvoltage}*SIR*FWR*NAC*FS\notag \\
    or, &\textit{Maximum per phase peak voltage}*SSR*SIR*FWR*NAC*FS\\
    &=\frac{132}{\sqrt{3}}*1.1*\sqrt{2}*2.8*1.2*1.15*1.1*1.2\notag \\
    &=604.6689 KV\notag
\end{align}
According to the table, number of discs required=6.\\[0.2cm]
Since the maximum number of insulators is 9 which will withstand all the overvoltage tests. Hence, the number of insulators required is 9.
\newpage
\section{Conductor Selection}
\subsection{Conductor Selection and Current Calculation}
The line current is calculated as follows:
\begin{align}
    \textit{Line Current(I)}&=\dfrac{P_{transfer}}{\sqrt{3}*cos\phi*V_{L-L}*N_c*N_b}
\end{align}
where, $N_b$ is the number of bundle in the given conductor which is taken as 1.
Hence, \
\begin{align*}
    or, I &=\dfrac{133}{\sqrt{3}*0.9*132*2*1}\\
    &=0.3231 KA \ or\  323.1 A
\end{align*}
Comparing this value of the current with the current carrying capacity from the given ACSR conductor table, conductor "CAT" is selected. Then, it must meet the efficiency criteria which must be greater than 94\% and the voltage regulation must be less than 12\% and the corona inception voltage must be greater than the maximum system voltage.
\subsubsection{Efficiency Criteria}
Initially, the \textbf{CAT} conductor of the current carrying capacity at $65^\circ$ is 323.1 A.\\
From the table, 
\[
\begin{aligned}
\text{Total cross sectional area(A)}=111.33\ mm^2\\
\text{Overall diameter(r)}=13.50mm\\
\text{Resistance at }20^\circ C(R_{20})=0.3077 \ \Omega/km
\end{aligned}
\]
For 110 km, \\
The coefficient of Resistivity($\alpha_{20}$)=$0.004/^\circ C$\\
The resistance at $65^\circ C$ is given by: 
\begin{align}
    R_{f}&=R_{0}(1+\alpha_{i}(\theta_f-\theta_i))\\
    &=0.3007(1+0.004(65-20))\notag \\
    &=0.354826 \ \Omega/Km\notag\\
    &=0.354826*110\notag \\
    &=39.03086 \ \Omega \notag
\end{align}
We know the transmission efficiency is given by: 
\begin{equation}
    Efficiency(\eta)=1-\frac{P_{loss}}{P_{transfer}}
\end{equation}
where, 
\begin{itemize}
    \item $P_{loss}=3 \times I^2\times R_{65}\times*N_c=3*323.1^2*39.03086*2=24447434.260W=24.474\ MW$
    \item $P_{transfer}=$ Power to be transferred=133\ MW
\end{itemize}
Thus, Transmission efficiency is : 
\begin{align*}
    &=1-\frac{24.447}{133}\\
    &=0.8182\ or\ 81.618\%
\end{align*}
Since, the transmission efficiency is less than required efficiency , we increase the current rating of conductor used and repeat it until we got the desired efficiency. The iterative process was done with efficiencies corresponding to conductors shown below:
% Table generated by Excel2LaTeX from sheet 'Sheet1'
% Table generated by Excel2LaTeX from sheet 'Sheet1'
\begin{table}[!htbp]
  \centering
  \caption{Efficiencies At Different Conductor Rating}
  \resizebox{0.5\textwidth}{!}{
    \begin{tabular}{|c|c|c|c|c|}
    \hline
    Conductor          & Ampacity(A)           & $R_{65^\circ C}(\Omega)$ & Loss(MW)           & $\eta\%$ \bigstrut\\
    \hline
    CAT                & 340                & 39.03086           & 24.44743426        & 81.618 \bigstrut\\
    \hline
    HARE               & 360                & 35.47434           & 22.21976649        & 83.293 \bigstrut\\
    \hline
    DOG                & 360                & 35.47434           & 22.21976649        & 83.293 \bigstrut\\
    \hline
    HYENA              & 360                & 35.00706           & 21.92708021        & 83.513 \bigstrut\\
    \hline
    LEOPARD            & 410                & 28.34832           & 17.75630077        & 86.649 \bigstrut\\
    \hline
    COYOTE             & 420                & 39.3943            & 24.67507914        & 81.447 \bigstrut\\
    \hline
    TIGER              & 420                & 28.58196           & 17.90264391        & 86.539 \bigstrut\\
    \hline
    WOLF               & 470                & 23.72744           & 14.86195871        & 88.826 \bigstrut\\
    \hline
    LYNX               & 520                & 20.45648           & 12.81315477        & 90.366 \bigstrut\\
    \hline
    PANTHER            & 560                & 17.69174           & 11.08142763        & 91.668 \bigstrut\\
    \hline
    LION               & 610                & 15.74474           & 9.861901483        & 92.585 \bigstrut\\
    \hline
    BEAR               & 650                & 14.18714           & 8.886280561        & 93.319 \bigstrut\\
    \hline
    GOAT               & 730                & 11.56518           & 7.243985343        & 94.553 \bigstrut\\
    \hline
    \end{tabular}}%
  \label{tab:Efficiencies}%
\end{table}\\
If the efficiency had not been greater than 94\% up to the very end of the table, we would have increased the number of bundles in the conductor and calculated the current and started the process again.\\[0.2cm]
Hence, we select the \textbf{GOAT} conductor moving forward.
\subsubsection{GMD and GMR calculation}
We know, for GOAT conductor:
\begin{itemize}
    \item Total sectional area(A)=399.98 $mm^2$
    \item Overall diameter($d_c$)=25.97 mm
    \item Overall radius(r)=12.985 mm
\end{itemize} 
The value of inductance can be calculated with the following formula:
\begin{equation}
    Inductance(L)=2*10^{-7}*ln\left(\frac{GMD}{GMR_L}\right)
\end{equation}
The value of capacitance can be calculated with the following formula: 
\begin{equation}
    Capacitance(C)=2*10^{-7}*\left(\frac{1}{ln\frac{GMD}{GMR_C}}\right)
\end{equation}
The calculation of GMD and GMR for a double circuit transmission line is as follows: 
\begin{figure}[!hb]
    \centering
    \includegraphics[width=0.5\textwidth,angle=-270]{Figures/GMDD.pdf}
    \caption{Three Phase Double Circuit Transmission Line}
    \label{fig: Double circuit Transmission Line}
\end{figure}\\
The vertical separation(y) and the horizontal separation between conductors(c) of each of the phases has been already been calculated above as y=3.6535m and c=8.3133m.
\[
\begin{aligned}
    D_{RY}=D_{YB}=D_{Y'R'}=D_{B'Y'}=y&=3.6535m\\
    D_{RB'}=D_{YY'}=D_{BR'}=c&=8.3133m\\
    D_{RR'}=D_{BB'}=\sqrt{(2y)^2+c^2}&=9.778m\\
    D_{YR'}=D_{RY'}=D_{BY'}=D_{YB'}=\sqrt{y^2+c^2}&=9.080m\\
    D_{RB}=D_{R'B'}=2y&=7.307m\\
\end{aligned}
\]
\textbf{GMD Calculation}\\[0.2cm]
The GMD of a three-phase double circuit line is given by: 
\begin{equation}
    GMD=\sqrt[3]{GMD_{RY}*GMD_{YB}*GMD_{BR}}
\end{equation}
where,
\begin{align*}
    GMD_{RY}=\sqrt[4]{D_{RY}*D_{RY'}*D_{R'Y}*D_{R'Y'}}&=5.7596m\\
    GMD_{YB}=\sqrt[4]{D_{YB}*D_{YB'}*D_{Y'B}*D_{Y'B'}}&=5.7596m\\[0.1cm]
\end{align*}
\begin{align*}
        GMD_{BR}=\sqrt[4]{D_{BR}*D_{BR'}*D_{B'R}*D_{B'R'}}&=7.7939m\\
\end{align*}
From equation 23:
\begin{align*}
    GMD=\sqrt[3]{5.7596*5.7596*7.7939}=6.370m
\end{align*}
\textbf{GMR Calculation}\\[0.2cm]
We know:
\begin{align*} 
    GMR_L&=r'=0.7788r=0.7788*0.012985m=0.0101m\\
    GMR_C&=r=0.012985m\\
\end{align*}
\end{document}